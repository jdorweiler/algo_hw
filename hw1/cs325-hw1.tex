\documentclass[a4paper,12pt]{article}
\usepackage[utf8]{inputenc}
\usepackage{listings}
\usepackage[margin=1in]{geometry}
\setlength\parindent{0pt}
\usepackage{xcolor}
\usepackage{graphicx}
\definecolor{dkgreen}{rgb}{0,0.6,0}
\definecolor{dred}{rgb}{0.545,0,0}
\definecolor{dblue}{rgb}{0,0,0.545}
\definecolor{lgrey}{rgb}{0.9,0.9,0.9}
\definecolor{gray}{rgb}{0.4,0.4,0.4}
\definecolor{darkblue}{rgb}{0.0,0.0,0.6}
\lstdefinelanguage{python}{
      backgroundcolor=\color{lgrey},  
      basicstyle=\footnotesize \ttfamily \color{black} \bfseries,   
      breakatwhitespace=false,       
      breaklines=true,               
      captionpos=b,                   
      commentstyle=\color{dkgreen},   
      deletekeywords={...},          
      escapeinside={\%*}{*)},                  
      frame=single,                  
      language=C++,                
      keywordstyle=\color{purple},  
      morekeywords={BRIEFDescriptorConfig,string,TiXmlNode,DetectorDescriptorConfigContainer,istringstream,cerr,exit}, 
      identifierstyle=\color{black},
      stringstyle=\color{blue},      
      numbers=right,                 
      numbersep=5pt,                  
      numberstyle=\tiny\color{black}, 
      rulecolor=\color{black},        
      showspaces=false,               
      showstringspaces=false,        
      showtabs=false,                
      stepnumber=1,                   
      tabsize=5,                     
      title=\lstname,                 
    }
    
%opening
\title{CS325 Homework 1}
\author{Kabir Kang, Paul Ely, Jason Dorweiler}

\begin{document}

\maketitle

\section{Section 1}



  \begin{lstlisting}[language=python,caption={pseudo code for $n^3$ algorithm}]
if len(array) == 1:
	maxSum = array[0]
else:
	for e in range(len(array)):
		for j in range(e,len(array)):
			maxSum = np.maximum(maxSum, sum(array[e:j]))
  \end{lstlisting}
  
  \begin{lstlisting}[language=python,caption={pseudo code for $n^2$ algorithm}]
for e in range(len(array)):
	testSum = 0
	for j in range(e,len(array)):
		testSum += array[j]
		maxSum = np.maximum(maxSum, testSum)
  \end{lstlisting}

  \begin{lstlisting}[language=python,caption={pseudo code for $n\log(n)$ algorithm}]
def algo3(array):
if(len(array) == 0):
	return 0
if(len(array) == 1):
	return array[0]

mid = len(array)/2
tempL = tempR = 0
maxLeft = maxRight = -99999

#left side crossing -- mid backwards
for i in range(mid,0,-1):
	tempL = tempL + array[i]
	maxLeft = np.maximum(maxLeft, tempL)

#right side crossing -- mid forwards
for j in range(mid+1, len(array)):
	tempR = tempR + array[j]
	maxRight = np.maximum(maxRight, tempR)
maxCrossing = maxLeft + maxRight

MaxA = algo3(array[:mid])
MaxB = algo3(array[mid+1:])

return np.maximum(np.maximum(MaxA, MaxB),maxCrossing))
  \end{lstlisting}  


\begin{figure}[h!]
\centering
\includegraphics[width=0.8\textwidth]{plotTo900}
\caption{Plot of the three algorithms up to array size of 900, top: log/log, bottom: normal axis}
\end{figure} 

\begin{figure}[h!]
\centering
\includegraphics[width=0.8\textwidth]{plotTo9000}
\caption{Plot of the three algorithms up to array size of 9000, top: log/log, bottom: normal axis}
\end{figure} 


\end{document}
