\documentclass[11pt,letterpaper]{article}
\usepackage[utf8]{inputenc}
\usepackage{amsmath}
\usepackage{amsfonts}
\usepackage{amssymb}
\usepackage[utf8]{inputenc}
\usepackage{listings}
\usepackage{float}
\usepackage[margin=1in]{geometry}
\setlength\parindent{0pt}
\usepackage{xcolor}
\usepackage{graphicx}
\definecolor{dkgreen}{rgb}{0,0.6,0}
\definecolor{dred}{rgb}{0.545,0,0}
\definecolor{dblue}{rgb}{0,0,0.545}
\definecolor{lgrey}{rgb}{0.9,0.9,0.9}
\definecolor{gray}{rgb}{0.4,0.4,0.4}
\definecolor{darkblue}{rgb}{0.0,0.0,0.6}
\lstdefinelanguage{python}{
      backgroundcolor=\color{lgrey},  
      basicstyle=\footnotesize \ttfamily \color{black} \bfseries,   
      breakatwhitespace=false,       
      breaklines=true,               
      captionpos=b,                   
      commentstyle=\color{dkgreen},   
      deletekeywords={...},          
      escapeinside={\%*}{*)},                  
      frame=single,                  
      language=C++,                
      keywordstyle=\color{purple},  
      morekeywords={BRIEFDescriptorConfig,string,TiXmlNode,DetectorDescriptorConfigContainer,istringstream,cerr,exit}, 
      identifierstyle=\color{black},
      stringstyle=\color{blue},      
      numbers=right,                 
      numbersep=5pt,                  
      numberstyle=\tiny\color{black}, 
      rulecolor=\color{black},        
      showspaces=false,               
      showstringspaces=false,        
      showtabs=false,                
      stepnumber=1,                   
      tabsize=5,                     
      title=\lstname,                 
    }
\author{Paul Ely, Jason Dorweiler, Kabir Kang}
\title{Project 3 - Linear Programming}
\begin{document}

\maketitle

\section*{Problem 1: mmmmm ... pork}

\paragraph{Objective.} We want to maximize the profits of a factory that produces hams, pork bellies, and picnic hams. Each of these products can be sold fresh or smoked. To solve this problem, consider the following variables: \\

\begin{tabular}{ll}
Variable & Definition \\
\hline
$H_F$ & Fresh hams \\
$H_R$ & Hams smoked on regular time \\
$H_O$ & Hams smoked on overtime \\
$B_F$ & Fresh pork bellies \\
$B_R$ & Pork bellies smoked on regular time \\
$B_O$ & Pork bellies smoked on overtime \\
$P_F$ & Fresh picnic hams \\
$P_R$ & Picnic hams smoked on regular time \\
$P_O$ & Picnic hams smoked on overtime \\
\end{tabular}

Our objective is to maximize the net profit (N) equation given by:
$N = 8 H_F + 14 H_R + 11 H_O + 4 B_F + 12 B_R + 7 B_R + 4 P_F + 13 P_R + 9 P_O$

\paragraph{Constraints}. The production of hams is subject to the following constraints:

\begin{itemize}
\item[] \textit{There are 480 hams, 400 pork bellies, and 230 picnic hams produced daily}. $H = 480$, $B = 400$, $P = 230$
\item[] \textit{Only 420 items can be smoked in regular time per day}. $H_R + B_R + P_R <= 420$
\item[] \textit{Only 250 items can be smoked in overtime.} $H_O + B_O + P_O <= 250$
\end{itemize}

\paragraph{Linear equation} The linear equation matrix is as follows: \\
\begin{align*}
max: 8 H_F &+ 14 H_R + 11 H_O \\
+ 4 B_F &+ 12 B_R + 7 B_R \\ 
+ 4 P_F &+ 13 P_R + 9 P_O
\end{align*}
\begin{align*}
s.t.: H_F + H_R + H_O &= 480 \\
B_F + B_R + B_O &= 400 \\
P_F + P_R + P_O &= 230 \\
H_R + B_R + P_R &<= 420 \\
H_O + B_O + P_O &<= 250
\end{align*}

\paragraph{Optimal solution.} We found the optimal solution to be: \\
\begin{tabular}{|c|c|cc|}
\hline 
& Fresh & Smoked (regular time) & Smoked (overtime) \\ 
\hline 
Hams & 440 & 0 & 40 \\ 
Pork belly & 0 & 400 & 0 \\ 
Picnic ham & 0 & 20 & 210 \\ 
\hline 
\multicolumn{3}{|c}{Total Net Profit:} & 10910.00 \\ 
\hline 
\end{tabular} 

\paragraph{Language/solver environment.} We used Python with the PuLP math package to solve the optimization problem. 

\begin{lstlisting}[language=python,caption={Code to solve linear program},mathescape]
//set up variables, minimum of 0 for each
HAM_FRESH = LpVariable("ham fresh", 0)
HAM_SRT = LpVariable("Ham Smoked RT", 0)
HAM_SOT = LpVariable("Ham Smoked OT", 0)

PORK_FRESH = LpVariable("PORK fresh", 0)
PORK_SRT = LpVariable("PORK Smoked RT", 0)
PORK_SOT = LpVariable("PORK Smoked OT", 0)

P_HAM_FRESH = LpVariable("P-ham fresh", 0)
P_HAM_SRT = LpVariable("P-Ham Smoked RT", 0)
P_HAM_SOT = LpVariable("P-Ham Smoked OT", 0)

// Create the 'prob' variable to contain the problem data
prob = LpProblem("Pork Profit", LpMaximize)

// objective to solve
prob += HAM_FRESH*8+HAM_SRT*14+HAM_SOT*11+PORK_FRESH*4+PORK_SRT*12+PORK_SOT*7+P_HAM_FRESH*4+P_HAM_SRT*13+P_HAM_SOT*9

// constraints
prob += HAM_FRESH+HAM_SRT+HAM_SOT <=480 # at most 480 ham
prob += PORK_FRESH+PORK_SRT+PORK_SOT <= 400 # at most 400 pork
prob += P_HAM_FRESH+P_HAM_SRT+P_HAM_SOT <= 230 # at most 230 picnic ham
prob += HAM_SRT+PORK_SRT+P_HAM_SRT <= 420 # max 420 smoked on RT
prob += HAM_SOT+PORK_SOT+P_HAM_SOT <= 250 # max 250 smoked on OT

// The problem data is written to an .lp file
prob.writeLP("porkprofit.lp")

// The problem is solved using PuLP's choice of Solver
prob.solve()

\end{lstlisting}

\end{document}